\documentclass[a4paper,11pt]{article}

\usepackage[left=2cm,text={17cm, 24cm},top=3cm]{geometry}
\usepackage[utf8]{inputenc}
\usepackage{times}
\usepackage[czech]{babel}

\begin{document}
\begin{center}
\Huge
\textsc{Vysoké učení technické v~Brně\\
}Fakulta informačních technologií\\
\vspace{\stretch{0.382}}
\LARGE Implementace interpretu imperativního jazyka IFJ16 \\
\Huge Tým 021, varianta a/3/I\\
\vspace{\stretch{0.309}}

\Large Vedoucí:	Kyzlink Jiří 	(xkyzli02)\\
				Kubiš Juraj		(xkubis15)\\
				Korček Juraj	(xkorce01)\\
				Kubica Jan		(xkubic39)\\
				Kovařík Viktor	(xkovar77)\\

\vspace{\stretch{0.309}}

\end{center}
{\Large \today \hfill
Brno}
\thispagestyle{empty}

\newpage

\tableofcontents

\newpage
\section{Úvod}
V této dokumentaci naleznete popis a návrh interpretu jazyka IFJ16, který je velmi zjednodušenou podmnožinou jazyka Java SE 8, což je staticky typovaný objektově orientovaný jazyk. Vybrali jsme si variantu varianta a/3/I, kde jsme měli za úkol přidat do interpretu vestavěnou funkci find, která využívala Knuth-Morris-Prattův algoritmus a funkci sort, kterou jsme měli implementovat tak, aby využívala shell sort.

--bude ještě doplněno-

\section{Syntaktický analyzátor}

\section{Sémantický analyzátor}

\section{Lexikální analyzátor}

\section{Interpret}

\section{Vestavěné funkce}
\subsection {vestavěné funkce v IAL.c}
--nevím jestli rozdělovat na podsekce ještě měnší nebo ne--
\subsubsection {find}
\subsubsection {sort}
\section{Testy}

\end{document}